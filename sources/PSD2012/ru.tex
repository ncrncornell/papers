%TCIDATA{Version=5.50.0.2960}
%TCIDATA{LaTeXparent=0,0,abowd-vilhuber-PSD-2012.tex}
                      

% $Id: ru.tex 396 2013-11-03 22:29:27Z lv39 $ 
% $URL: https://forge.cornell.edu/svn/repos/ncrn-cornell/branches/papers/PSD2012/ru.tex $                      


\subsection{Linear Mixed Models}

We produced $R-U$ (Risk-Utility) curves or $R-U$ confidentiality maps that
examine the trade-off between $\epsilon $ (disclosure risk) and correlations
(data utility) by changing parameter values in our procedure. Duncan \cite%
{duncan} states that \textquotedblleft in its most basic form, an $R-U$
confidentiality map is the set of paired values $(R,U)$ of disclosure risk
and data utility that correspond to various strategies for data
release.\textquotedblright\ In our models, $\epsilon $ changes to generate
the $R-U$ curve and lower values of $\epsilon $ correspond to lower levels
of risk and higher levels of privacy. As $\epsilon $ decreases, the privacy
of our released data increases as defined by $\epsilon $-differential
privacy. Low disclosure risk has good differential privacy, which says that
\textquotedblleft any possible outcome of an analysis should be
\textquotedblleft almost\textquotedblright\ equally likely, independent of
whether any individuals opts into or opts out of the data
set\textquotedblright\ \cite{dwork}. In addition, since the Laplace scale
parameter is $\frac{\Lambda }{k\epsilon }$, the random noise added to
release $\epsilon $-differentially private data increases as $\epsilon $
decreases ($k$ increases at a slower rate than $\epsilon $ decreases). This
means that the released data or estimates are more noisy for lower values of 
$\epsilon $. Consequently, data utility should be lower for released data
with more noise added. We examined the exact trade-off between disclosure
risk, $\epsilon $, and data utility, correlation $(\hat{y}^{DP_{\epsilon }}$,%
$y)$.

\subsubsection{$R-U$ Curve for Linear Mixed Models}

{For all values of $\epsilon $, calculate the predicted rates: 
\begin{equation*}
JCR^{DP}=\hat{y}^{DP_{\epsilon }}=X\hat{\beta}^{DP_{.51\epsilon }}+Z\hat{u}%
^{DP_{.49\epsilon }} 
\end{equation*}%
For $k=1$ or all of the data, calculate the predicted rates: 
\begin{equation*}
JCR^{global}=\hat{y}^{global}=X\hat{\beta}^{global}+Z\hat{u}^{global} 
\end{equation*}%
Calculate the correlations between $y$ and $\hat{y}^{global}$, $\hat{y}%
^{DP_{\epsilon }}.$ Finally, plot the correlations as a function of $%
\epsilon $.}

\subsubsection{$R-U$ Curve for Linear Models}

{For all values of $\epsilon $, calculate the predicted rates: 
\begin{equation*}
JCR^{DP_{\epsilon }}=\hat{y}^{DP_{\epsilon }}=X\hat{\beta}^{DP_{\epsilon }} 
\end{equation*}%
For $k=1$ or all of the data, calculate the predicted rates: 
\begin{equation*}
JCR^{global}=\hat{y}^{global}=X\hat{\beta}^{global} 
\end{equation*}%
Calculate the correlations between $y$ and $\hat{y}^{global}$, $\hat{y}%
^{DP_{\epsilon }}$. Finally, plot the correlations as a function of $%
\epsilon $. \ The LM only estimated industry means and did not include a
time trend.}

Figures \ref{Fi:ru1} and \ref{Fi:ru2} show the R-U Curves for the LMMs and
LMs, respectively. Correlations decreased as $\epsilon $ decreased, and all
correlations of $\hat{y}^{DP_{\epsilon }}$ with $y$ were lower than the
global \textquotedblleft best fit\textquotedblright\ correlation when $k=1$
(which would correspond to non-differentially private $\epsilon >25$). Since
all correlations including the one between $y$ and $\hat{y}^{global}$ were
less than 0.40, the model did not fit the data well. This illustrates the
principle limitation of the differentially private estimator -- more random
effects were required to get a good fit, detailed industry and time effects
in particular, but such models were only feasible when $\epsilon >25,$ which
is no protection at all. But for models with approximately 3,000 effects,
degradation in correlation over decreased values of $\epsilon $ was only
slightly noticeable. Non-monotonicity was observed when most of the noise
was added to $\beta $ versus $u$ since there were only 21 random Laplace
draws.

\subsubsection{R-U Curve for Linear Mixed Models with Allocated Privacy}

Additionally, we considered having proportionally different levels of
privacy for $\beta $ and $u$ within the total privacy budget of $\epsilon $.
Since there were many more estimates of $u$ (3,161) as compared to industry $%
\beta $ (20), it may be reasonable to protect the estimates of $u$ with more
privacy (lower $\epsilon $). Figures \ref{Fi:ru3} and \ref{Fi:ru4} show $u$
having 10\% and 88\% of the plotted value of $\epsilon $, respectively,
while $\beta $ accounts for the remainder. For example, in Figure \ref%
{Fi:ru3}, the five budgets of $\epsilon $ used for $u$ were $%
0.46,0.4,0.3,0.2,$ and $0.1$ while the budgets used for $\beta $ were $%
4.05,3.52,2.64,1.76,$ and $0.88$. \ Figures \ref{Fi:ru5} and \ref{Fi:ru6}
show $u$ having 1\% and 97\% of the plotted privacy value of $\epsilon $,
respectively. Noticeable degradation is seen in Figure \ref{Fi:ru5} when $u$
is highly protected. \ For all figures except Figure \ref{Fi:ru2}, the
privacy budget of the time trend was kept at 2\%.

\subsubsection{An Improved LMM and Influential Observations}

We also examined the effects of deleting all of a county's $U$ observations
on the estimates of variance components and fixed effects with a LMM\ that
included four parameters for lagged quarterly rates. The base fit of the
model improved significantly to a correlation of 49.67\% as compared to just
under 35\% for the LMM\ including only a simple time trend. The goal was
also to bound the possible leave-1-out changes of our REML\ estimates, $\hat{%
\beta},\hat{\sigma}_{\xi }^{2},$and $\hat{\sigma}_{uc}^{2}$ for closer
inspection for both the LMM\ and BLMM. We performed over three thousand
leave-U-out simulations for each county. For each simulation, the process is
as follows:

\begin{itemize}
\item Define $D^{\symbol{126}U}\equiv (y^{\symbol{126}U},X^{\symbol{126}%
U},Z^{\symbol{126}U})$, differing by one county-industry combination (U-out);

\item Fit REML\ estimates and analyze changes in $\hat{\beta},\hat{\sigma}%
_{\xi }^{2},$and $\hat{\sigma}_{uc}^{2}.$
\end{itemize}

Results for the leave U-out fixed effects indicate that all industry
estimates are within 0.0002 of each other except for public administration,
which has a range of 0.003. The covariates for the lagged quarterly rates
were all within 0.001 of each other. The 0.1 and 99.9 percent quantiles for
the variance components are described in Table \ref{Ta:varcomp}.

\begin{table}[h]
\begin{center}
\begin{tabular}{|c|c|c|c|}
\hline
Variance Component & MLE & 0.1\% quantile & 99.9\% quantile \\ \hline
$\hat{\sigma}_{\xi }^{2}$ & 0.01045409 & 0.01043703 & 0.01046005 \\ \hline
$\hat{\sigma}_{uc}^{2}$ & 0.00016302 & 0.00015722 & 0.00016330 \\ \hline
\end{tabular}%
\end{center}
\caption{Variance Estimate Ranges from Leaving out One County}
\label{Ta:varcomp}
\end{table}

Results from the analysis of deleting influential observations from Section
V indicate that all updated estimates of fixed effects and variance
components are well within the bounds of the leave U-out changes. The county
EBLUPs that were most affected by the removal of influential observations
were always the particular counties that these observations were in. The
maximum change for the EBLUPs was 0.007828 (observation from county 3047)
and the industry fixed effects was 0.00008281 (observation from county 661).
Both of these observations came from observations with large $\hat{u}$'s and
large absolute values of $r_{i}^{p}$. If we were to match these maximum
changes correspond to four times the standard deviation of a Laplace random
variable, they produce Laplace scale parameters of 0.0015 and 0.000016,
respectively. To put things in perspective, the laplace scale parameter for
the estimated fixed effects and EBLUPs in the sub-sample and aggregate
approach when $\epsilon $ was unity was approximately 0.0002 and was $%
\epsilon $ was 3 was approximately 0.00012. With no sub-sampling and the
removal of an influential observation, the laplace noise would only protect
the fixed effects. \ Results for the BLMM\ focus on the county random
effects.

\subsection{Bayesian Linear Mixed Models}

We analyzed the implications of the removal of influential observations on
the $\epsilon $-differential privacy of our county random effects according
to Section V-B.2. Predictably, in those models deleting observations from
small counties (3047 and 661) produced the largest proportional bin changes
across all models. Each model had 62,220 bins corresponding to 20 bins for
each of the 3,111 county random effects. The model deleting an observation
from county 3047 had as few as 21 posterior samples in its smallest bin
(3,217 in its largest) and the model deleting an observation from county 661
had 3,458 posterior samples in its largest bin (26 in its smallest). Both of
these unusual counts occurred in the county effect from which the
influential observation was deleted. Comparing these results to the
benchmark model with 500 observations in each bin and using the methodology
developed in Section V-B.2, this corresponds to an overall $\epsilon $ of
3.2.

We compared these results to random noise which is represented by the
replicated benchmark model that was fit to monitor convergence in Section
V-B.3. The bin boundaries were fixed at the five percent quantiles from the
complete-data estimation. Hence, the expected count in each bin is 500. The
replicated model using the complete data had its smallest bin containing 382
posterior samples and its largest bin containing 641 posterior samples. This
corresponds to an overall $\epsilon $ of 0.27, which is illustrated in the
histogram of the bin counts shown in Figure \ref{Fi:histogram} where the
mode is 500, the distribution is symmetrical, and the minimum and maximum on
the horizontal axis define the inputs to computing $\epsilon $. Since no
rows have been excluded from this experiment, the interpretation of $%
\epsilon $ is the deviation in the empirical differential privacy that
results from the imprecision of using 10,000 posterior samples.

The 32 models with deleted influential observations always had maximum and
minimum bin counts between the extremes of the replicated benchmark model
and the models with deleted observations from county 3047 or county 661.
That is, the extreme values used to estimate $\epsilon $ empirically came
from the values computed when influential observations were deleted from
these one of these two counties. A histogram of the bin counts for the model
deleting an influential observation from county 3047, which defined the
overall $\epsilon $ of 3.2, is shown in Figure \ref{Fi:histogramcounty3047}.

%%% Local Variables:
%%% mode: latex
%%% TeX-master: abowd-schneider-vilhuber-JPC-2012.tex
%%% End:
