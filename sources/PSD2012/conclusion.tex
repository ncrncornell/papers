%TCIDATA{Version=5.50.0.2960}
%TCIDATA{LaTeXparent=0,0,abowd-vilhuber-PSD-2012.tex}
                      

% $Id: conclusion.tex 396 2013-11-03 22:29:27Z lv39 $
% $URL: https://forge.cornell.edu/svn/repos/ncrn-cornell/branches/papers/PSD2012/conclusion.tex $

In the United States, the Confidential Information Protection and
Statistical Efficiency Act of 2002 (CIPSEA) formalized the obligation of
every federal statistical agency to take long-term custody of the
confidential micro-data used for its work. These agencies all face the same
problem as the U.S. Census Bureau, which assumed a comparable obligation
when U.S. Code Title 13 was adopted in 1954 and national statistical
agencies around the world, which usually operate under legal constraints
that forbid granting long-term custody to an entity that is not part of
their government. The acquisition, archival and curation system described
here can be generalized to restricted-access research requirements of many
statistical agencies and private data stewards. The tools would allow such
agencies to harness the efforts of researchers who want to understand the
structure and complexity of the confidential data they intend to analyze in
order to propose and implement reproducible scientific results. Future
generations of scientists can build on those efforts because the long-term
data preservation operates on the original scientific inputs, not inputs
that have been subjected to statistical disclosure limitation or
privacy-preserving filters prior to entering the repository. Such curation
provides sponsors like national scientific research organizations with a
viable system for enforcing data management plans on projects, ensuring that
results can be tested now and replicated many years in the future.
