%TCIDATA{Version=5.50.0.2960}
%TCIDATA{LaTeXparent=0,0,abowd-vilhuber-PSD-2012.tex}
                      

% $Id: introduction.tex 396 2013-11-03 22:29:27Z lv39 $
% $URL: https://forge.cornell.edu/svn/repos/ncrn-cornell/branches/papers/PSD2012/introduction.tex $

The era of public-use micro-data as a cornerstone of empirical research in
the social sciences is coming to an end---not because it is no longer
feasible to create such data without breaching confidentiality. It still is,
and statistical agencies like the Census Bureau will continue to do so.
Rather, the death knell is being sounded by young scholars pursuing research
programs that mandate inherently identifiable data: geospatial relations,
exact genome data, networks of all sorts, linked administrative records, and
so on. These researchers acquire authorized restricted access to the
confidential, identifiable data and perform their analyses in secure
environments. And their research is challenging fundamental scientific
principles because the restricted access cannot be extended arbitrarily to
the whole user community \cite{Huberman2012}.

The Census Research Data Centers are a leading paradigm for such research,
but other modalities are proliferating rapidly. The researcher is allowed to
publish results that have been filtered through a statistical disclosure
limitation protocol. Scientific scrutiny is hampered because the researcher
cannot effectively implement a data management plan that permits sharing
these restricted-access data with other scholars. In the case of Census RDCs
the relevant statute has been interpreted to prohibit granting long-term
data custody outside of the Bureau except for copies held by the National
Archives, which does not permit public access to these holdings.
University-operated archives like ICPSR may take custody of non-Census
Bureau restricted-access data under some conditions, but they still cannot
freely grant access to the confidential micro-data in their repositories.
The data custody problem is impeding the \textquotedblleft acquire, archive
and curate\textquotedblright\ model that dominated social science data
preservation in the era of public-use micro-data.
