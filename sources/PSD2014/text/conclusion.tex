In this paper, we have described two alternate mechanisms to substitute for suppressions in  small-cell tabulations of business 
microdata, with the goal of improving analytic validity while maintaining a sufficiently high 
standard of disclosure limitation. Neither mechanism fundamentally changes the existing suppression methodology, rather, the mechanisms work to fill in the holes created by the suppression methodology. 

Leveraging the availability of a high-quality synthetic datasets 
(the Synthetic LBD) with proven disclosure limitation efficiency and analytic validity \cite{KinneyEtAl2011}, the first 
method is very simple, but may suffer from seam biases and time-inconsistency. The second 
method aims to improve on that by ``blending in'' synthetic establishments, which may slightly 
reduce analytic validity in time periods where the strict application of the suppression 
algorithms would no longer impose any constraints, but improving on the time-series properties 
of the released data. 

Several limitations of the research presented here should be highlighted. The examples provided in this article rely on an earlier release of the Synthetic LBD 
\cite{KinneyEtAl2011}. Recent developments to improve the micro-level analytic validity of the 
\ac{SynLBD} \cite{CES-WP-2014-12} should improve the analytic validity of the mechanisms 
proposed here as well.  We also compare our proposed mechanisms to the actual published, but otherwise unmodified \ac{BDS}. Comparing to post-publication improvements to a table with suppressions \cite{HolanEtAl2010} will inevitably lead to an apparent reduction in the utility of this particular approach. Finally, the approach relies on continuous availability of synthetic microdata with analytical validity. Other approaches rely on fewer data points, and thus be favored due to lower implementation costs.

Future work for this paper involves assessing the procedure on a wider variety of variables, better synchronisation of the computational algorithms underlying the BDS and the SynBDS, and improved assessment at the microdata level of the protection afforded by Algorithm~1.





